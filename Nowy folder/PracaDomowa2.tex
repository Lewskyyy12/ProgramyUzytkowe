\documentclass[a4paper, 12pt]{amsart}
\usepackage{amsmath}
\usepackage[T1]{fontenc}
\usepackage[polish]{babel}
\author{Imie Nazwisko}
\title{Wprowadzanie w tryb matematyczny}
\begin{document}
\maketitle

\section{SYMBOLY MATEMATYCZNY}
\subsection{Sumy, iloczyny i całki. $ \sum_{k=1}^{n} , \prod_{k=1}^{n} , \int_{0}^{\frac{\pi}{2}} $} 
\paragraph{ Dla każdego $ n \in \mathbb{N}$ spełniona jest równość}
$$ \sum_{k=1}^{n} k^2 = \frac{n(n+1)(2n+1)}{6} .$$
\paragraph{Także $\forall n \in \mathbb{N},$ spełnia się}
$$ \prod_{k=2}^{n+1} \left(1-\frac{1}{k^2}\right)=\frac{2n+1}{2n}. $$
Wewnątrz akapitu suma może być napisana jako\(\sum_{k=1}^{n}a^n\) albo jako \({\sum\limits_{k=1}^{n}a^n}\), iloczyn jako 
\(\prod_{k=1}^{n} a^n \) albo \(\prod\limits_{k=1}^{n} a^n \).
\subsection{Funkcje Eulera}
\begin{equation} \Gamma(z)= \int\limits_{0}^{+\infty}t^{z-1}e^{-t}dt
\end{equation}
Drugim sposobem określenia funkcji $\Gamma$ (dla dowolnych liczb zespolonych jest:
\begin{equation} \Gamma(z)= \lim_{{n \to +\infty}} \frac{n!n^z}{z(z+1)(z+2)...(z+n)}=\frac{1}{z}\prod\limits_{n=1}^{\infty}
\frac{\left(1+\frac{1}{n}\right)^z}{1+\frac{z}{n}}.
\end{equation}
Możemy także określić odwrotność funkcji Gamma następująco($\gamma$ to stała Eulera-Mascheroniego):
\begin{equation} \frac{1}{\Gamma(z)}=ze^{\gamma z} \prod\limits_{n=1}^{\infty}\left[\left(1+
\frac{z}{n}\right)e^{-\frac{z}{n}}\right]. 
\end{equation}
Wzór (1) jest definicją \textbf{Funkcji Eulera.}

\end{document}
