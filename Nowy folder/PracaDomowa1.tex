\documentclass[a4paper,10pt]{article}
\title{\textbf{PRACA DOMOWA N\textsuperscript{\underline{o}}1}}
\author{{\scriptsize IMIE NAZWISKO}}
\usepackage[T1]{fontenc}
\usepackage{hyperref}
\begin{document}
\sloppy
\maketitle
\item{1.} Przykład tekstu po-angielsku: Aesop "The Hare and the Tortoise" \hfill{}
\newline  \par
 One day the Hare laughed at the short
feet and slow speed of the Tortoise. The
Tortoise replied:\par 
\it ''You may be as fast as the wind, but
I will beat you in a race''\par
\rm
The Hare thought this idea was impossible and he agreed to the proposal.
It was agreed that the Fox should choose
the course and decide the end.\par
The day for the race came, and the
Tortoise and Hare started \textbf{together.}\par
\textbf{The Tortoise never stopped for
a moment}, walking slowly but steadily,
right to the end of the course. The Hare
ran fast and stopped to lie down for a
rest. But he fell fast asleep. Eventually,
he woke up and ran as fast as he could.
But when he reached the end, he saw the
Tortoise there already, sleeping comfortably after her effort.
\begin{center} 2. Rozmiary czcionki 
\end{center}
\par My przeczytaliśmy wszystkie polecenia na stronie:\par
\mbox{\url{https://www.overleaf.com/learn/latex/Font_sizes\%2C_families\%2C_and_styles.} }
\par Teraz możemy to wykorzystać.\par
{\tiny Czcionka jest tutaj bardo mała, trzeba powiększyć
ten tekst}.{\footnotesize Teraz już lepiej, ale jeszcze nie
zwykle.} O, tutaj jest normalny tekst!
{\large A tutaj jest większa czcionka dla
czegoś specjalnego.}{\huge I, oczywiście, teraz spróbujemy my największej
czcionki}


\end{document}